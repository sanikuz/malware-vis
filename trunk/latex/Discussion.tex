\label{sec:discuss}
This section discusses the project collaboration process, the feedbacks and ideas for potential future work.

\textbf{Collaboration.} This project has involved close collaboration between students at the graphics$\&$visualization and the information security centers. Initially, the visualization design and the data collection processes were conducted separately. However, to build a visualization model that addresses both utility and aesthetic concerns requires close collaboration among participants. Hence in the development phase, participants met and discussed issues with data processing, visualization and user interaction on a regular basis. Convenient design and collaboration was made possible through shared repository of open source interfacing tools~(\cite{jython}~\cite{dpkt}) and the Processing visualization library~\cite{Processing}. 

\textbf{Feedbacks.} We found building MalwareVis to be a rewarding experience: not only the participants with different backgrounds have learned from each other, but the users are intrigued by our design as well during informal user testing. From their feedbacks, we have concluded two major advantages of the cell view. The first advantage is that it provides a global view: a single cell view is able to represent up to 6000 flow records in a pcap file without significant visual clutter~(Fig.~\ref{fig:samples}). The second advantage is that it provides a graphical user interface that allows the user to analyze different malware behaviors interactively.

\textbf{Future Development.} During our informal user testing, a user asked if there is any relationship between different malware instances and how to visualize them. Binary and n-ary relationships do exist within a corpus of malware instances. For example, malware frequently found in honeypots actually behave as downloaders and simply download, install and execute other malicious binaries. Furthermore, malware is often packed resulting in binaries that are MD5 unique, but exhibit identical behavior. Currently, MalwareVis does not support highlighting these relationships automatically, but we believe this is a possible direction for future work. 
 



