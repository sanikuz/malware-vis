\label{sec:prior}
Most of the related network visualization systems so far has been traffic-centric with explicit focuses on each record of flows. For example, the work of McPherson and Ma~\cite{McPherson:2004} presents PortVis, which visualizes large network flow data collected at an Internet gateway. Their system allows interactive timeline filtering and histogram summarization on port activities. NVisionIP~\cite{Lakkaraju:2004}, introduced by Lakkaraju and his colleagues, focuses on the static visual representation of the IP network traffic in a 2D scatter plot named galaxy view. The other two views~(the small multiple view and the machine view) complement the main view by providing bar charts of various flow counts. TNV by Goodall et al.~\cite{Goodall:2005} aims to visualize the source, destination and the time attributes in a single view. To allow the user to explore the area of interest in greater detail, TNV can zoom in on traffic through bifocal display techniques. Rumint~\cite{Conti} is another full-fledged, time-oriented network traffic analyzer that captures live traffic and supports playing back the traffic.

Apart from scatter plots histograms and bar charts, linked graphs have become popular representations for network flows. Based on force-directed graph layout techniques, Afterglow~\cite{Marty} is a collection of utilities for generalized communication network visualization. Recently, Blue and his colleagues presented NetGrok~\cite{Blue:2008} that consists of graph and treemap views. In their graph view, the IP layer communications with the local host are visualized as a star network with other hosts organized around based on their IP addresses.
Although using graphs for security visualization has increased in popularity,
it is unclear to the authors how to incorporate the time-series nature of network
flow data into graphs without using a complementary view.

Compared to previous traffic-centric visualization systems, the major
innovation and advantage of MalwareVis lies in its entity-based design. The
idea is to visualize the network traces associated with each malware instance
as a whole. It is natural to represent the overall flow associated with each
sample as one model for the purpose of identifying potentially interesting
patterns. For example, Quist and Liebrock visualize the compiled executables
by monitoring malicious program execution~\cite{Quist11}. While our work
focus on malware network traces, the purpose is similar---to allow the analysts
to identify particular patterns of network traffic. Based on this idea, we
design a intuitive visualization model, the cell view, to present behaviors of
malware network communications.

In the scope of visualizing large volumes of NetFlow data, Fischer et al.\cite{Fischer:2008} presents a visual analytics system for monitoring and visualizing large networks using TreeMap; FloVis~\cite{Taylor:2009} presents a suite of visualization tools that display various aspects of multiple individual host behaviors as color-coded time series and show the interactions among hosts and groups of hosts.






