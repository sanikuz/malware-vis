\label{sec:interaction}
In this section, we present the user interface of MalwareVis. 
\begin{figure*}
	\centering
		\includegraphics[width = 0.88\linewidth]{pics/Screen.png}
	\label{fig:Screen}
	\caption{MalwareVis's user interface has a table view that allows the user to browse and filter malware entities and the cell view for graphical representations, interaction and comparision.}
\end{figure*}

To start, the user can select a set of pcap files (typically a corpus of malware network traces) and MalwareVis parses them into a collection of entities~(Fig.~\ref{fig:Overview}). Parsing and preprocessing takes approximately 30 seconds for 95 pcap files~(average file size 722KB). In addition to reading raw pcap files, the user can select a set of previously saved entities for quickly initiating the application. 

\textbf{Table View: } As shown in Fig.~\ref{fig:Screen}, the program then generates table views. The \emph{pcap table} displays the entity digests, each of which is a summary of a list of communication sessions recorded by a pcap file. The user can browse the list of entities and search or filter them based on MD5 values and duration. The lower part of the table view is the \emph{stream table} that list a set of streams of a user-selected entity. 

\textbf{Single Cell View:} After the user select a entity, she can create a cell view of the entity by clicking the ``single'' button in the cell view control panel. A cell view appears in a very small fraction of a second. In the cell view, the user can interact with a string of cilia: the user can select a cilium by clicking the oval head of the cilium. Then the disk panel shows the detailed stream information represented by this cilium. It also queries a GeoIP database to identify the country name and flag of the host IP if it is a routable IP address. The user can also select from the stream table and its corresponding stream and cilium are highlighted in both views. 

\textbf{Timeline:} The user can change the value of $\alpha$ in the timeline mapping algorithm~(Alg.~\ref{alg:mapping}) from the slider in the cell view control panel. Notice that when sliding, the cell view redraws the set of cilia at each frame. This creates intermediate frames between the two views of different $\alpha$ values~(Fig.~\ref{fig:Tune}). Therefore, the user can view the layout equalization processing on the timeline as a animation.   

\textbf{Multiple Cell Views:} The user can also select multiple malware entities. Then, stacked cell views of a list of the selected entities can be created by clicking the ``multiple'' button in the cell view control panel. The user can compare the visual representations of multiple malware samples~(Fig.~\ref{fig:samples}) and interact with each view as one would interact with a single cell view. 

 